\documentclass[journal]{vgtc}                % final (journal style)
%\documentclass[review,journal]{vgtc}         % review (journal style)
%\documentclass[widereview]{vgtc}             % wide-spaced review
%\documentclass[preprint,journal]{vgtc}       % preprint (journal style)
%\documentclass[electronic,journal]{vgtc}     % electronic version, journal

%% Uncomment one of the lines above depending on where your paper is
%% in the conference process. ``review'' and ``widereview'' are for review
%% submission, ``preprint'' is for pre-publication, and the final version
%% doesn't use a specific qualifier. Further, ``electronic'' includes
%% hyperreferences for more convenient online viewing.

%% Please use one of the ``review'' options in combination with the
%% assigned online id (see below) ONLY if your paper uses a double blind
%% review process. Some conferences, like IEEE Vis and InfoVis, have NOT
%% in the past.

%% Please note that the use of figures other than the optional teaser is not permitted on the first page
%% of the journal version.  Figures should begin on the second page and be
%% in CMYK or Grey scale format, otherwise, colour shifting may occur
%% during the printing process.  Papers submitted with figures other than the optional teaser on the
%% first page will be refused.

%% These three lines bring in essential packages: ``mathptmx'' for Type 1
%% typefaces, ``graphicx'' for inclusion of EPS figures. and ``times''
%% for proper handling of the times font family.

\usepackage{mathptmx}
\usepackage{graphicx}
\usepackage{times}
\usepackage{epstopdf}
%% We encourage the use of mathptmx for consistent usage of times font
%% throughout the proceedings. However, if you encounter conflicts
%% with other math-related packages, you may want to disable it.

%% This turns references into clickable hyperlinks.
\usepackage[bookmarks,backref=true,linkcolor=black]{hyperref} %,colorlinks
\hypersetup{
  pdfauthor = {},
  pdftitle = {},
  pdfsubject = {},
  pdfkeywords = {},
  colorlinks=true,
  linkcolor= black,
  citecolor= black,
  pageanchor=true,
  urlcolor = black,
  plainpages = false,
  linktocpage
}

%% If you are submitting a paper to a conference for review with a double
%% blind reviewing process, please replace the value ``0'' below with your
%% OnlineID. Otherwise, you may safely leave it at ``0''.
\onlineid{0}

%% declare the category of your paper, only shown in review mode
\vgtccategory{Research}

%% allow for this line if you want the electronic option to work properly
\vgtcinsertpkg

%% In preprint mode you may define your own headline.
%\preprinttext{To appear in IEEE Transactions on Visualization and Computer Graphics.}

%% Paper title.

\title{ZFP and CUDA}

%% This is how authors are specified in the journal style

%% indicate IEEE Member or Student Member in form indicated below
\author{Mark Kim, Peter Lindstrom, and Charles Hansen}
\authorfooter{
%% insert punctuation at end of each item
\item
 Mark Kim is with SCI Institute at the University of Utah. E-mail: mbk@cs.utah.edu.
\item
 Peter Lindstrom is with Lawrence Livermore National Laboratory. E-mail:  lindstrom@llnl.gov
\item
 Charles Hansen is with SCI Institute at the University of Utah. E-mail: hansen@cs.utah.edu.
}

%other entries to be set up for journal
\shortauthortitle{Biv \MakeLowercase{\textit{et al.}}: Global Illumination for Fun and Profit}
%\shortauthortitle{Firstauthor \MakeLowercase{\textit{et al.}}: Paper Title}

%% Abstract section.
\abstract{High performance computing has seen a remarkable shift over the
last ten years as GPGPU has become intrinsic in the design of large supercomputers to achieve improved scaling performance over traditional CPU-only systems. Unfortunately, as GPU accelerator performance has increased non-linearly, memory bandwidth over the PCI bus has stagnated which has resulted in increased latency. Further, the size of GPU RAM has increased without a corresponding increase in memory bandwidth, which increases the fixed time cost of moving data between main memory and the GPU.

Without increasing the physical memory bandwidth, software solutions are required if we are to achieve exascale computing. Previously, a fixed rate floating point compressor, zfp, was introduced. This lossy scheme usually compresses to an accuracy within machine epsilon delta. Unfortunately, a GPU implementation was unavailable. Therefore, we introduce a GPGPU implementation of zfp.

} % end of abstract

%% Keywords that describe your work. Will show as 'Index Terms' in journal
%% please capitalize first letter and insert punctuation after last keyword
\keywords{Compression, GPGPU}

%% ACM Computing Classification System (CCS). 
%% See <http://www.acm.org/class/1998/> for details.
%% The ``\CCScat'' command takes four arguments.

%\CCScatlist{ % not used in journal version
% \CCScat{K.6.1}{Management of Computing and Information Systems}%
%{Project and People Management}{Life Cycle};
% \CCScat{K.7.m}{The Computing Profession}{Miscellaneous}{Ethics}
%}

%% Uncomment below to include a teaser figure.
%   \teaser{
%    \centering
%    \includegraphics[width=16cm]{CypressView}
%   \caption{In the Clouds: Vancouver from Cypress Mountain.}
%  }

%% Uncomment below to disable the manuscript note
%\renewcommand{\manuscriptnotetxt}{}

%% Copyright space is enabled by default as required by guidelines.
%% It is disabled by the 'review' option or via the following command:
% \nocopyrightspace

%%%%%%%%%%%%%%%%%%%%%%%%%%%%%%%%%%%%%%%%%%%%%%%%%%%%%%%%%%%%%%%%
%%%%%%%%%%%%%%%%%%%%%% START OF THE PAPER %%%%%%%%%%%%%%%%%%%%%%
%%%%%%%%%%%%%%%%%%%%%%%%%%%%%%%%%%%%%%%%%%%%%%%%%%%%%%%%%%%%%%%%%

\begin{document}

%% The ``\maketitle'' command must be the first command after the
%% ``\begin{document}'' command. It prepares and prints the title block.

%% the only exception to this rule is the \firstsection command
\firstsection{Introduction}

\maketitle

%% \section{Introduction} %for journal use above \firstsection{..} instead

High performance computing has seen a remarkable shift over the
last ten years as GPGPU has become intrinsic in the design of large supercomputers to achieve improved scaling performance over traditional CPU-only systems. Unfortunately, as GPU accelerator performance has increased non-linearly, memory bandwidth over the PCI bus has stagnated which has resulted in increased latency. Further, the size of GPU RAM has increased without a corresponding increase in memory bandwidth, which increases the fixed time cost of moving data between main memory and the GPU.

Without increasing the physical memory bandwidth, software solutions are required if we are to achieve exascale computing. Previously, a fixed rate floating point compressor, zfp, was introduced. This lossy scheme usually compresses to an accuracy within machine epsilon. Unfortunately, a GPU implementation was unavailable to alleviate the bottleneck on the GPU. Therefore, we introduce a GPGPU implementation of zfp.

\section{Previous Works}
Something about previous works.
\subsection{Nebo}
Some stuff about Nebo.

\subsection{ZFP on the CPU}
ZFP on CPU.

\section{GPU}
\subsection{Encode}
Similar to the CPU encoding, the GPU encoding is divided into different components. First, the block is transformed into a fixed-point format by computing the maximum exponent for a block of floating point values and then the floating-point value is converted to a fixed-point integer block of values (Sec.~\ref{subsubsec:fixed}). In Section~\ref{subsubsec:forwardxform}, the block of integers are lifted to decorrelate the integers. Then, in Section~\ref{subsubsec:reorder}, the integer block is converted to an unsigned integer and finally in Sec.~\ref{subsubsec:encode}, the unsigned block is encoded. 

\subsubsection{Fixed-Point Conversion}
\label{subsubsec:fixed}

First, the maximum exponent of a 4x4x4 block is computed. For every floating-point value in the block, the maximum exponent of the block is computed. To parallelize this in CUDA, a CUDA thread is assigned to each block and the every cell in the block is traversed to determine the maximum exponent. 

Once the maximum exponent of the floating point block is determined,
all the floating-point values of the block are converted to fixed-point representation by normalizing to the maximum exponent 
and casting this value to an integer. Similar to the maximum exponent, for every block, a CUDA thread is assigned and the thread loops over all the blocks in the thread to convert the floating-point values to integers.

\subsubsection{Forward Transform}
\label{subsubsec:forwardxform}
To achieve good compression, spatially similar values need to be decorrelated. A separable transform is used as in Lindstrom~\cite{lindstrom:2014}. Using the transform matrix for a 4-vector, 

\begin{flushleft}
	\begin{equation}
	A = \frac{1}{2}\left( \begin{array}{llll}
	1 &1 &1 &1 \\
	c &s &-s &-c\\
	1 &-1 &-1 &1\\
	s &-c &c &-s
	
	\end{array}\cfrac \right)
	\label{eq:cotenergy}
	\end{equation}
\end{flushleft}
and
\begin{flushleft}
	\begin{equation}
	\begin{array}{lr}
	s = \sqrt{2}sin(\frac{\pi}{2}t) &
	c = \sqrt{2}cos(\frac{\pi}{2}t)
	
	\end{array}
	\label{eq:cotenergy}
	\end{equation}
\end{flushleft}

where $t \in [0,1]$
\subsubsection{reorder}
\label{subsubsec:reorder}
\subsubsection{Encode}
\label{subsubsec:encode}

\section{Results}
\section{Future Works}

\begin{enumerate}
\item use real world test data.
\item Re-factor the CUDA code to get more performance. This is on-going.
\item Test performance with and without zfp.
\end{enumerate}

%% if specified like this the section will be committed in review mode
\acknowledgments{
The authors wish to thank A, B, C. This work was supported in part by
a grant from XYZ.}

\bibliographystyle{abbrv}
%%use following if all content of bibtex file should be shown
%\nocite{*}
\bibliography{template}
\end{document}

